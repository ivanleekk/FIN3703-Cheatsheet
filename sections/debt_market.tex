\section{Debt Market}
Bonds are promissory notes that make fixed payments to the holder

Bond prices and interest rates are inversely related, and the price of a bond is the present value of future cash flows.

$Price = \frac{C}{1+r} + \frac{C}{(1+r)^2} + \ldots + \frac{C + FV}{(1+r)^n}$

Where $C$ is the coupon payment, $r$ is the interest rate, $FV$ is the face value of the bond, and $n$ is the number of periods

Yield to Maturity (YTM) is the internal rate of return of a bond, and is the discount rate that makes the present value of the bond's cash flows equal to the price of the bond

Longer maturity bonds have higher interest rate risk, and lower coupon bonds have higher interest rate risk, obvious as they have longer duration.

Bonds with lower coupon rates are more sensitive to interest rate changes, and bonds with higher coupon rates are less sensitive to interest rate changes as low coupon means more of the value is held by the face value.

\subsection{Types of Bonds}

\begin{tabularx}{\linewidth}{X X X}
\toprule
\textbf{Type} & \textbf{Description} & \textbf{Example} \\
\midrule
Perpetual Bonds & Bonds with no maturity date & Consol \\
\midrule
Zero-Coupon Bonds & Bonds with no coupon payments & Treasury Bills \\

\midrule
Callable Bonds & Bonds that can be redeemed by the issuer before maturity & Corporate Bonds \\
\midrule
Convertible Bonds & Bonds that can be converted into equity & Convertible Bonds \\
\midrule
Floating Rate Bonds & Bonds with interest rates that change over time & Floating Rate Notes based on SOFR or SORA \\
\midrule
Domestic Bonds & Bonds issued in the country's currency & USD Bond in USA by US Firm \\
\midrule
Eurobonds & Bonds issued in a foreign currency & USD Bond issued in Europe \\
\midrule
Foreign Bonds & Bonds where the currency $\neq$ issuer nationality & SGD Bond issued by a US Firm \\

\bottomrule
\end{tabularx}

\subsection{Yield Curves}
Graphical relationship between the yield and maturity of bonds of the same credit rating, and the shape of the yield curve can predict future economic conditions

\begin{callout}
	The yield curve is normally constructed with T-Bills and T-Bonds, as they have low default risk and low liquidity risk.
\end{callout}

Yield curves can be upward sloping, flat, or inverted, and the slope of the yield curve can be used to predict future interest rates

\begin{tabularx}{\linewidth}{X X}
\toprule
\textbf{Yield Curve} & \textbf{Economic Condition} \\
\midrule
Upward Sloping & Economy is growing, and interest rates are expected to rise \\
\midrule
Flat & Economy is slowing down, and interest rates are expected to fall \\
\midrule
Inverted & Economy is in recession, and interest rates are expected to fall \\
\bottomrule
\end{tabularx}

\subsection{Types of US Treasuries}

Treasury Bills (T-Bills) are short-term securities with maturities of 1 year or less, and are sold at a discount to face value

Treasury Notes (T-Notes) are medium-term securities with maturities of 2 to 10 years, and pay semi-annual coupon payments

Treasury Bonds (T-Bonds) are long-term securities with maturities of 20 to 30 years, and pay semi-annual coupon payments

Treasury Strips are Seperate Trading of Registered Interest and Principal of Securities, and are zero-coupon bonds created by stripping the interest and principal payments of T-Notes and T-Bonds

\subsection{Types of Singapore Government Securities}

Singapore Government Securities (SGS) are issued by the Monetary Authority of Singapore (MAS) and are used to manage liquidity in the banking system

Difference is SGS are issued to meet bank's needs for a risk free asset rather than to finance government expenditure

Primary Market of SGS is by competitive and non competitive bidding. However there is a 40\% limit for non competitive bidding, and competitive bids will be allocated on order from lowest to higest yields, then all bids will be allotted the same higest accepted yield.

Singapore Savings Bonds (SSB) are issued by the Singapore Government and are a special type of SGS that are designed for retail investors, and are issued monthly

Singapore Corporate Debt market is growing, most are issued in USD and SGD
