\section{Derivatives}
A derivative is a financial instrument whose value is derived from the value of an underlying asset, rate, or index.

It helps in the price discovery process, hedging, and speculation.

Forwards and futures give an indication of the value of the underlying asset at a future date.

Forwards are traded over-the-counter (OTC) and are customised to the needs of the buyer and seller.

Futures are standardised contracts traded on exchanges.

Options give the holder the right, but not the obligation, to buy or sell an asset at a predetermined price.

Swaps are agreements to exchange cash flows or assets at a future date and are traded OTC.

\subsection{Forward \& Futures}

Forward is a bilateral agreement between two counterparties to trade a specified quantity of a specified good at a specified price and date in the future.

Notably:
\begin{itemize}
	\item Bilateral because it is customised to the needs of the buyer and seller.
	\item Customisable because the contract can be tailored to the needs of the buyer and seller.
	\item There is default risk because the contract is between two parties.
	\item Non-negotiable because neither party can transfer the contract to another party.
\end{itemize}

Futures are standardised contracts traded on exchanges. A clearing house acts as the counterparty to both buyer and seller, reducing default risk. Margin accounts are used to manage default risk. Either party can close out its position by reversing it.

\subsubsection{Margin Accounts}
Initial margin is the amount of money that must be deposited by the buyer and seller to cover potential losses.

Mark to Market is the process of adjusting the margin account to reflect the current value of the contract.

Maintenance margin is the minimum amount of money that must be maintained in the margin account.

Margin effectively is posting collateral to cover potential losses and prevent default.

The margin level is important for liquidity as having too high a margin level can reduce liquidity in the market, but too low can increase default risk.

\subsubsection{Hedging}
Hedging is the process of reducing risk by taking an offsetting position in the market.

Long hedge is when the buyer of the asset takes a long position in the futures market to hedge against an increase in the price of the asset.

Short hedge is when the seller of the asset takes a short position in the futures market to hedge against a decrease in the price of the asset.

\subsubsection{Basis Risk}
Basis risk is the risk that the price of the asset in the spot market and the futures market will not move in the same direction.

$B_t = S_t - F_t$, where $B_t$ is the basis at time $t$, $S_t$ is the spot price at time $t$, and $F_t$ is the futures price at time $t$.

\subsubsection{Delivery Options}
Since futures are standardised contracts, there are delivery options that specify how the contract will be settled.

In practice, the delivered grade is usually the cheapest to deliver grade.


\subsection{Options}
Options give the holder the right, but not the obligation, to buy or sell an asset at a predetermined price.

Call options give the holder the right to buy an asset at a predetermined price.

Put options give the holder the right to sell an asset at a predetermined price.

European options can only be exercised at the expiration date.

American options can be exercised at any time before the expiration date.


