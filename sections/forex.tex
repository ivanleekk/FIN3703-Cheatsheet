\section{Forex}
Forex is an OTC market with dealers that stand ready to buy and sell

Major FX trading centers: London, New York, Tokyo, Singapore, Hong Kong facillitate 78\% of exchange trading

Turnover of \$7.5 Trillion per day in April 2022

24/7 market with regional peak trading times.

In Singapore, total monthly FX volume is $\approx$ \$20 Trillion USD

\subsection{International Role of the USD}
Used as a reserve currency, and a store of value.

However the share of USD in foreign reserves is lowering, and currencies like the RMB are increasing in importance.

Most USD Reserves are held in the form of US Treasury Securities.

\subsection{Exchange Rate Quotations}
Base Currency: The currency that is being quoted against another currency (The first in the pair)

Quote Currency: The currency that is being quoted against the base currency (The second in the pair)

For examples, in the pair USD/SGD, USD is the base currency and SGD is the quote currency. If the rate is 1.35, it means 1 USD is worth 1.35 SGD.

Bid-Ask Spread is similar to market makers in equity markets. The bid is the price at which the dealer will buy the base currency, and the ask is the price at which the dealer will sell the base currency.

\subsection{Cross Rates}
Cross rates is the exchange rate between two currencies that are not trading directly against each other.

E.g. if USD/AUD is 1.3718 --- 1.3728 and USD/GBP is 0.7608 --- 0.7616, then GBP/AUD is $\frac{\text{USD/AUD}}{\text{USD/GBP}} = \frac{1.3718}{0.7616}$ --- $\frac{1.3728}{0.7608}$. Since you are buying GBP at the ask and sell Aak at the Bid and vice versa

2 point arbitrage is conducting simultaneous transactions in two or more markets to take advantage of price discrepancies. Such as if a currency is overvalued in USA, but undervalued in AUS or others.

Arbitrage can only happen if there is a Ask price lower than the Bid price of the same pair in a different market. i.e to say no overlap at all in the range of bid-ask spreads.

High Inflation generally leads to currency depreciation as the purchasing power of the currency decreases.

US Dollar Index (DXY) is a measure of the value of the US dollar relative to a basket of foreign currencies such as the Euro (\euro) , Swiss Franc, Japanese Yen(\textyen), Canadian Dollar, British Pound (\textsterling) and Swedish Krona.


\subsection{Parity Conditions}
\subsubsection{PPP}
Absolute PPP $S_{B/A} = \frac{P_B}{P_A}$, where $S_{B/A}$ is the spot exchange rate, $P_B$ is the price of the basket of goods in country B and $P_A$ is the price of the basket of goods in country A.

PPP implied exchange rate is the exchange rate that would make the price of the basket of goods in two countries equal.

By comparing PPP implied exchange rate with the actual exchange rate, we can determine if a currency is overvalued or undervalued. E.g. based on Big Mac Index, USD/SGD = 1.1687, but actual exchange rate is 1.3376, therefore SGD is undervalued against USD by the market by 12.626\%.

Other way is to compare changes in price level and use that to reflect changes in exchange rate.

For example, if the inflation rate in the US is 3\% and in Singapore is 2\%, then the exchange rate should change by 1\%. $\frac{S_{B/A, t}}{S_{B/A, t - 1}} = \frac{1+\pi_A}{1+\pi_B}$

There are some factors that cause PPP to not hold such as transportation costs, tariffs, taxes, non-tradable goods and services (Healthcare, Property, \ldots), etc.

\subsubsection{Interest Rate Parity}
Covered Interest Rate Parity (CIP) states that the difference in interest rates between two countries should be equal to the difference in the spot exchange rate and the forward exchange rate.

$\frac{f_{B/A}}{S_{B/A}} = \frac{(1+r_A)}{(1+r_B)}$

Where $f_{B/A}$ is the forward exchange rate, $S_{B/A}$ is the spot exchange rate, $r_A$ is the interest rate in country A and $r_B$ is the interest rate in country B.

Note: the horizons of the forward exchange rate and the interest rate should be the same.

For example, if $S_{USD/GBP} = 0.8465, r_{USD} = 2\%, r_{GBP} = 4\%$, then the forward exchange rate should be $0.8465 \times \frac{1.02}{1.04} = 0.8302$, which results in non-arbitrage between spot and forward FX rates.

This should prevent any arbitrage of the interest rate differential between two countries.

Covered Interest Arbitrage (CIA) is when the parity is violated and there is an opportunity to make riskless profit by bringing it back to CIP Condition

If $S_{(USD/GBP)} = 0.8465, i_{US} = 2\%, i_{GB} = 4\%$, then CIP implied \textbf{three-month} forward rate f(USD/GBP) equals to $0.8465\times\frac{1 + 4\%/4}{1+2\%/4} = 0.8507$

If three-month forward rate in the market ($f_{market}$) is now quoted at 0.8482, an arbitrageur would sell USD at spot (S) and buy with the forward market quote ($f_{market}$).

\subsection{Singapore Exchange Rate Regime}
Normally there are two main monetary policy instruments: Domestic Interest Rate and Exchange Rate.

However Singapore uses the exchange rate as the main monetary policy instrument. Due to the small size of the economy and the openness of the economy, the exchange rate is used to control inflation and maintain economic stability.

MAS therefore keeps the exchange rate in a specified policy band. To preserve the purchasing power of the SGD.

SGD is managed against a basket of currencies of Singapore's major trading partners and competitors.

The band has three components: the slope (appreciation/inflation rate), the width (volatility) and the center (nominal effective exchange rate).


