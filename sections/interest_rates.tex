\section{Interest Rates}


\begin{tabularx}{\linewidth}{X X X}
\toprule
\textbf{Aspect} & \textbf{Nominal Interest Rate} & \textbf{Real Interest Rate} \\
\midrule
Definition & The interest rate before adjusting for inflation & The interest rate after adjusting for inflation \\
\midrule
Formula & Nominal Rate $\approx$ Real Rate + Inflation Rate & Real Rate $\approx$ Nominal Rate - Inflation Rate \\
\midrule
Impact of Inflation & Does not account for inflation & Accounts for inflation \\
\midrule
Example & If nominal rate is 5\% and inflation is 2\%, the real rate is 3\% & If nominal rate is 5\% and inflation is 2\%, the real rate is 3\% \\
\bottomrule
\end{tabularx}

Fisher Equation is used to account for tax

$[1+i(1-t)] = [1+r(1-t)](1+p)$

Where $i$ is the nominal interest rate, $r$ is the real interest rate, $t$ is the tax rate, and $p$ is the inflation rate

\subsection{Types of Inflation}
Headline Inflation is the total inflation in an economy, while Core Inflation is the inflation excluding food and energy prices

\begin{callout}
    In Singapore, core inflation excludes accommodation and private road transport, known as MAS Core Inflation
\end{callout}

Expected Inflation brings forward purchasing, investment in properties, demand for higher wages, and fuels actual inflation

Deflation is the opposite of inflation, where prices fall, and is bad for the economy as consumers will delay purchases, 
and businesses will delay investments, past debts will become more expensive, and the real inflation rate is too high
(use the real interest rate formula to visualise)

Stagflation is when there is high inflation, high unemployment, and low economic growth, and is caused by supply-side shocks. In this case,
both monetary and fiscal policy will be ineffective and hurt the economy.

\subsection{Effective vs Quoted Interest Rates}
Different loan replayment schemes have different effective interest rates, and the effective interest rate is the true cost of borrowing

Annualised Interest Rate normally refers to APR, Interest Rate per period $\times$ Number of periods in a year


\begin{tabularx}{\linewidth}{X X X}
\toprule
\textbf{Type} & \textbf{Formula} & \textbf{Example} \\
\midrule
Quoted Rate & $r = \frac{I}{P}$ & 5\% \\
\midrule
Effective Rate & $r_{eff} = (1 + \frac{I}{P})^n - 1$ & 5.12\% \\
\bottomrule
\end{tabularx}

Amortisation is the process of paying off a loan, and the loan is paid off in equal instalments, with the interest decreasing over time, and the principal increasing over time

The periodic payment and proportion of interest and principal depends on the way interest is quoted, such as Declining Balance Bases, Annual Rest Basis or Flat Basis

Declining Balance is calculating interest at every payment cycle, aka monthly Rest

Annual Rest is calculating interest based on the remaining principal at the start of the payment year

Flat Basis is calculating interest based on the original principal, with no change in interest vs principal over time

