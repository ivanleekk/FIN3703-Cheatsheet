\section{Intro to Financial Markets}
% \subsection{Basic Definitions}
% \begin{itemize}
%     \item Financial assets are legal claims to future benefits which have monetary values
%     \item Financial markets are structures where financial assets are traded
%     \item Financial institutions are institutions or business entities which provide financial services
% \end{itemize}

\subsection{Assets}
Tangible assets are physical assets

Intangible assets are non-physical assets such as legal claims to future benefits

Roles of financial assets:
\begin{itemize}
    \item Helps in fulfiling the functions of finance
    \item Efficient pooling of funds
    \item Efficient distribution of ownership with Limited Liability
    \item Management of risk
\end{itemize}

\subsection{Types of Financial Assets}
\begin{itemize}
    \item Debt
    \begin{itemize}
        \item Bank Loans
        \item Government Bonds
        \item Corporate Bonds (Commercial Papers)
    \end{itemize}
    \item Equity
    \begin{itemize}
        \item Common Stock
        \item Preferred Stock
    \end{itemize}
    \item Derivatives (Options, Futures, Swaps)
    \item Convertible Securities (Convertible Bonds, Convertible Preferred Stock)
\end{itemize}

\subsection{Properties of Financial Assets}
    Money is a medium of exchange and can be a cash or cheque
    Near Money is a liquid asset that can be quickly converted to cash

    \begin{callout}
        Under MAS, M1 = currency in active circulation, M2 = M1 + Near Money
    \end{callout}
    Financial assets can have different divisibility
    \begin{itemize}
        \item Deposits, basically infinite divisible
        \item Bonds, usually in \$1000 denominations
        \item Stocks, usually in 100 shares (lots)
    \end{itemize}
    Reverserbility or round-trip cost is the cost of buying and selling an asset
    \begin{itemize}
        \item Bid-Ask Spread from Market Makers, from our perspective, we buy at Ask and sell at Bid
        \item Commissions from Brokers
        \item Stamp Duties / Taxes
        \item Loadings (sales charges on buying or redeeming mutual funds)
    \end{itemize}
    Term to Maturity is the time left to the maturity of the asset
    \begin{itemize}
        \item Short-term assets (T-Bills)
        \item Long-term assets (Bonds)
        \item Perpetual assets [Equities / Perpetuities / Consols (Government Perpetuities)]
    \end{itemize}
    Liquidity is the ease of converting an asset to cash
    Can be seen as the loss due to \textbf{IMMEDIATE} sale of the asset

    Risk is the uncertainty of an event and can be measured using standard deviation ($\sigma$), 
    variance ($\sigma^2$), or beta ($\beta$) etc.

    Tax Status is the tax treatment of the asset
    \begin{callout}
        Some countries like Singapore do not have Capital Gains or Dividends Tax, which 
        can be a benefit compared to other countries
    \end{callout}

\subsection{Financial Markets}
Financial markets are structures where financial assets are traded, moving funds from lender-savers to borrowers-spenders

\subsection{Roles of Financial Markets}
\begin{itemize}
    \item Provision of liquidity by gathering willing buyers and sellers
    \item Efficient pricing of assets, by providing a market to facilitate price discovery
    \item Reduction of transaction costs, by providing a centralised location for trading, reducing search time, contracting cost and counterparty risk
\end{itemize}

\subsection{Types of Financial Markets}
Primary Markets are markets where new securities are issued
Such as IPOs, Rights Issues, Private Placements, SEOs

Secondary Markets are markets where existing securities are traded

Markets can also be classified by the type of assets traded, such as Debt, Equity, Foreign Exchange or Derivatives

Money Markets are markets for short-term debt securities, usually less than 1 year
Such as T-Bills, Commercial Papers, Bankers Acceptances

Money Markets also provide for short-term borrowing between banks
\begin{itemize}
    \item Discount Window, where banks can borrow from the Central Bank
    \item Interbank Market, where banks can borrow from each other using Fed Funds to meet reserve requirements
    \item Repurchase Agreements (Repo), where banks can borrow from each other using securities as collateral
     that are sold and later repurchased at a specified time and price
\end{itemize}

Capital Markets are markets for long-term debt and equity securities, usually more than 1 year
Such as Corporate Bonds, Common Stock, Preferred Stock

Exchange Traded Markets are markets where trading is done on a centralised exchange
OTC Markets are markets where trading is done over-the-counter, usually between dealers directly and price is not publicly disclosed (Has higher counterparty risk)

Spots are prices for immediate delivery
Futures or Forwards are prices for future delivery